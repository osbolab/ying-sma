Configure the makefile by invoking {\ttfamily cmake} in the build directory. Do not invoke {\ttfamily cmake} in any component's root directory or any other path. \begin{DoxyVerb} # -DCMAKE_BUILD_TYPE=Debug - attach symbols to the output (on by default)
 # -Dbuild_tests=OFF        - don't build test executables

 cd core/build/
 cmake ../
 make
 ./libsma_test
\end{DoxyVerb}


Project sources are declared in {\ttfamily C\-Make\-Lists.\-txt}.

\subsection*{Testing}

Using gtest\-: \begin{DoxyVerb}test/test1.hh
------------------
#pragma once

#include "log.hh"
#include "gtest/gtest.h"

TEST(MyClass, it_does_x_correctly)
{
  LOG(DEBUG) << "Hello!";
  ASSERT_EQ(1, 1);
  ASSERT_GT(1, 0);
  LOG(FATAL) << "Something terrible happened";
}

test/test_main.cc
------------------
/**************************************************************************
 * This file just creates a compilation unit with the tests.

#include "log.hh"
// Must come first and only once in application 
_INITIALIZE_EASYLOGGINGPP

#include "test1.hh"
// more tests

#include "gtest/gtest.h"

int main(int argc, char** argv)
{
  // Allow command line arguments to specify tests
  ::testing::InitGoogleTest(&argc, argv);

  // Trim down some noisy output
  // Optionally turn logging off or change debug level (default ALL)
  el::Loggers::reconfigureAllLoggers(
    el::ConfigurationType::Format,
    "%datetime{%m:%s.%g} %levshort [%thread] %func (%fbase:%line) %msg");

  return RUN_ALL_TESTS();
}
\end{DoxyVerb}


\subsection*{Project Setup}

\subsubsection*{Using C++11}

\paragraph*{N\-S-\/3 via {\ttfamily wscript}}

\begin{DoxyVerb}def build(bld):
  obj = bld.create_ns3_program('ex', ['csma', 'internet', 'applications'])
  obj.cxxflags = ['-std=c++11']
  obj.source = ['ex.cc']
\end{DoxyVerb}


Compile and run by changing into the {\ttfamily ns-\/3.\-xx/build} directory and running {\ttfamily ./waf -\/-\/run example}

\paragraph*{Android}

\begin{quotation}
If you're using command line N\-D\-K support this is what I had to put in my {\ttfamily jni/\-Application.\-mk} to get C++11 support with A\-P\-I 19

\end{quotation}


(\href{http://stackoverflow.com/a/21386866}{\tt stackoverflow}) \begin{DoxyVerb}NDK_TOOLCHAIN_VERSION := 4.8
# APP_STL := stlport_shared  --> does not seem to contain C++11 features
APP_STL := gnustl_shared

# Enable c++11 extentions in source code
APP_CPPFLAGS += -std=c++11
\end{DoxyVerb}


\subsection*{Protocols}

\subsubsection*{Content dissemination}


\begin{DoxyEnumerate}
\item Interest a. Announcement a. Replication
\end{DoxyEnumerate}
\begin{DoxyEnumerate}
\item Content a. Announcement a. Request a. Distribution a. Caching
\end{DoxyEnumerate}

\paragraph*{Interest}

\subparagraph*{Announcement}

Some consumer {\itshape G} generates an interest {\itshape I\-\_\-\-T} in content type {\itshape T}. She broadcasts this interest to her nearest neighbors, and they in turn broadcast the interest if they have not already.

Each node that sees this interest inspects her {\itshape active interest table} (A\-I\-T) for a match on {\itshape T}. The A\-I\-T maps {\itshape T} to some future time point.

If she finds an entry in her A\-I\-T she increments the mapped time by the interest expiry duration {\itshape t}; if she finds no entry then she adds it with an initial value of {\itshape t}.

Once every node in the network has seen {\itshape I\-\_\-\-T} and has an entry in her A\-I\-T for {\itshape T}{\ttfamily -\/$>$}$\ast$t$\ast$ the interest is {\itshape saturated} and the announcement phase is complete.

\subparagraph*{Replication}

When a node arrives to the network he announces his arrival to his nearest neighbors. Each of them responds by broadcasting their interest tables to him until he has a complete table. He uses the mapped time values stored by the other nodes and does not increment them.

\paragraph*{Content}

Some producer {\itshape P} generates a content {\itshape C\-\_\-\-T} of type {\itshape T} and computes its hash {\itshape H}. She segments this content into {\itshape N} blocks {\itshape L} bytes long and indexes them from 0 to {\itshape N}-\/1. She then names each block {\itshape B\-\_\-i} by concatenating the hash value of the original content with the block's index and hashing that concatenated value. Finally she stores these blocks in her local cache, thus becoming a {\itshape seed} for {\itshape C\-\_\-\-T}.

\subparagraph*{Announcement}

When the producer {\itshape P} has finished generating the blocks for her content she will notify her immediate neighbors that she has new content with type {\itshape T} and that its name is {\itshape H}, the original content's hash value.

\subparagraph*{Request}

\subparagraph*{Distribution}

\subparagraph*{Caching}